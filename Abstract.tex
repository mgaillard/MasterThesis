\thispagestyle{plain}

\section*{Abstract}

In this master thesis, we present our study on Convolutional Neural Networks Features and Perceptual Hashing for Large Scale Reverse Image Search. We especially focus our attention on robustness of such systems against common modifications (Gaussian blur, color filter, resize, compression, rotation, cropping). In a first part, we design a benchmark to evaluate the speed and accuracy of several existing techniques. These techniques have very good retrieval performances except against rotation and cropping. In a second part, we investigate the use of CNN Features for reverse image search. Experiments show that they are considerably robust against modifications. To efficiently perform a nearest neighbor search we advocate the use of hashing into short binary codes. In a third part, we propose a supervised method for learning a binary hash function that preserve similarity. This method is based on LSH with random projection and Minimal Loss Hashing, we propose a new approach to optimize the hash function in a continuous space. Experiments show that this approach is valid and promising for hashing CNN Features.

\section*{R\'esum\'e}

Dans cette th\`ese de master, nous pr\'esentons une \'etude sur les caract\'eristiques extraites \`a partir de r\'eseaux de neurones convolutifs et les hash perceptuels pour la recherche d'images invers\'ee \`a grande \'echelle. Nous portons en particulier notre attention sur la robustesse de ces syst\`emes face \`a des modifications communes (flou Gaussien, filtre de couleur, redimensionnement, compression, rotation, rognage). Dans un premier temps, nous concevons un protocole pour comparer la vitesse et la pr\'ecision de plusieurs techniques existantes. Ces techniques ont de bonnes performances, except\'ees contre la rotation et le rognage. Dans un second temps, nous examinons l'utilisation des caract\'eristiques extraites \`a partir de r\'eseaux de neurones convolutifs pour la recherche d'image invers\'ee. Les exp\'eriences montrent qu'elles sont consid\'erablement robustes aux modifications. Pour effectuer efficacement une recherche de plus proches voisins, nous recommandons l'utilisation de techniques de hachage binaire. Dans un troisi\`eme temps, nous proposons une m\'ethode supervis\'ee pour apprendre une fonction de hachage binaire qui pr\'eserve les similarit\'es. Cette m\'ethode est bas\'ee sur LSH avec des projections al\'eatoires et sur Minimal Loss Hashing. Nous proposons une nouvelle approche pour optimiser la fonction de hachage dans un espace continu. Les exp\'eriences montrent que cette approche est valide et prometteuse pour hacher les caract\'eristiques extraites \`a partir de r\'eseaux de neurones convolutifs.